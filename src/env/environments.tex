% Template:     Template Tesis LaTeX
% Documento:    Definición de entornos
% Versión:      2.2.0 (22/04/2021)
% Codificación: UTF-8
%
% Autor: Pablo Pizarro R.
%        Facultad de Ciencias Físicas y Matemáticas
%        Universidad de Chile
%        pablo@ppizarror.com
%
% Sitio web:    [https://latex.ppizarror.com/tesis]
% Licencia MIT: [https://opensource.org/licenses/MIT]

\newenvironment{references}{
\ifthenelse{\equal{\stylecitereferences}{bibtex}}{
	}{
		\throwerror{\references}{Solo se puede usar entorno references con estilo citas \noexpand\stylecitereferences=bibtex}
	}
	\phantomsection
	\addcontentsline{toc}{chapter}{\namereferences}
\begin{thebibliography}{}
\ifthenelse{\equal{\bibtextextalign}{justify}}{
		}{
		\ifthenelse{\equal{\bibtextextalign}{left}}{
			\raggedright
		}{
		\ifthenelse{\equal{\bibtextextalign}{right}}{
			\raggedleft
		}{
		\ifthenelse{\equal{\bibtextextalign}{center}}{
			\centering
		}{
			\throwbadconfig{Ajuste de linea referencias bibtex desconocido}{\bibtextextalign}{justified,left,right,center}}}}
		}
	}
	{
	\end{thebibliography}
}
\newenvironment{anexo}{
\begingroup
	\phantomsection
\changeheadertitle{\nomltappendixsection}
\def\GLOBALsectionalph{false}
	\bookmarksetup{
		numbered={true},
		openlevel={\thetemplateBookmarksLevelPrev}
	}
	\appendixtitleon
	\appendixtitletocon
	\bookmarksetupnext{level=part}
\begin{appendices}
		\ifthenelse{\equal{\showappendixsecindex}{true}}{}{
\pdfbookmark{\nameappendixsection}{appendix}
		}
		\ifthenelse{\equal{\appendixindepobjnum}{true}}{
\counterwithin{equation}{chapter}
\counterwithin{figure}{chapter}
\counterwithin{lstlisting}{chapter}
\counterwithin{table}{chapter}
		}
	}{
	\end{appendices}
\def\GLOBALsectionalph{false}
\bookmarksetupnext{level={\thetemplateBookmarksLevelPrev}}
	\bookmarksetup{
		numbered={\cfgpdfsecnumbookmarks},
		openlevel={\cfgbookmarksopenlevel}
	}
	\endgroup
}
\newcommand{\coreinitsourcecodep}[4]{
	\emptyvarerr{\coreinitsourcecodep}{#2}{Estilo de codigo no definido}
	\checkvalidsourcecodestyle{#2}
	\ifthenelse{\equal{\showlinenumbers}{true}}{
		\rightlinenumbers}{
	}
	\lstset{
		backgroundcolor=\color{\sourcecodebgcolor}
	}
	\ifthenelse{\equal{\codecaptiontop}{true}}{
		\ifx\hfuzz#4\hfuzz
			\ifx\hfuzz#3\hfuzz
				\lstset{
					escapeinside={(*@}{@*)},
					style=#2
				}
			\else
				\lstset{
					escapeinside={(*@}{@*)},
					style=#2,
					#3
				}
			\fi
		\else
			\ifx\hfuzz#3\hfuzz
				\lstset{
					caption={#4 #1},
					captionpos=t,
					escapeinside={(*@}{@*)},
					style=#2
				}
			\else
				\lstset{
					caption={#4 #1},
					captionpos=t,
					escapeinside={(*@}{@*)},
					style=#2,
					#3
				}
			\fi
		\fi
	}{
		\ifx\hfuzz#4\hfuzz
			\ifx\hfuzz#3\hfuzz
				\lstset{
					escapeinside={(*@}{@*)},
					style=#2
				}
			\else
				\lstset{
					escapeinside={(*@}{@*)},
					style=#2,
					#3
				}
			\fi
		\else
			\ifx\hfuzz#3\hfuzz
				\lstset{
					caption={#4 #1},
					captionpos=b,
					style=#2
				}
			\else
				\lstset{
					caption={#4 #1},
					captionpos=b,
					escapeinside={(*@}{@*)},
					style=#2,
					#3
				}
			\fi
		\fi	
	}
}
\lstnewenvironment{sourcecodep}[4][]{
	\coreinitsourcecodep{#1}{#2}{#3}{#4}
}{
	\ifthenelse{\equal{\showlinenumbers}{true}}{
		\leftlinenumbers}{
	}
}
\newcommand{\importsourcecodep}[5][]{
	\coreinitsourcecodep{#1}{#2}{#3}{#5}
	\inputlisting{#4}
	\ifthenelse{\equal{\showlinenumbers}{true}}{
		\leftlinenumbers}{
	}
}
\newcommand{\coreinitsourcecode}[3]{
	\emptyvarerr{\coreinitsourcecode}{#2}{Estilo de codigo no definido}
	\checkvalidsourcecodestyle{#2}
	\ifthenelse{\equal{\showlinenumbers}{true}}{
		\rightlinenumbers}{
	}
	\lstset{
		backgroundcolor=\color{\sourcecodebgcolor}
	}
	\ifthenelse{\equal{\codecaptiontop}{true}}{
		\ifx\hfuzz#3\hfuzz
			\lstset{
				escapeinside={(*@}{@*)},
				style=#2
			}
		\else
			\lstset{
				escapeinside={(*@}{@*)},
				caption={#3 #1},
				captionpos=t,
				style=#2
			}
		\fi
	}{
		\ifx\hfuzz#3\hfuzz
			\lstset{
				escapeinside={(*@}{@*)},
				style=#2
			}
		\else
			\lstset{
				escapeinside={(*@}{@*)},
				caption={#3 #1},
				captionpos=b,
				style=#2
			}
		\fi
	}
}
\lstnewenvironment{sourcecode}[3][]{
	\coreinitsourcecode{#1}{#2}{#3}
}{
	\ifthenelse{\equal{\showlinenumbers}{true}}{
		\leftlinenumbers}{
	}
}
\newcommand{\importsourcecode}[4][]{
	\coreinitsourcecode{#1}{#2}{#4}
	\lstinputlisting{#3}
	\ifthenelse{\equal{\showlinenumbers}{true}}{
		\leftlinenumbers}{
	}
}
\newenvironment{itemizebf}[1][]{
	\begin{itemize}[font=\bfseries,#1]
	}{
	\end{itemize}
}
\newenvironment{enumeratebf}[1][]{
	\begin{enumerate}[font=\bfseries,#1]
	}{
	\end{enumerate}
}
\newenvironment{resumenenv}[4]{
	\clearpage
	\emptyvarerr{\resumenenv}{#1}{Tabla resumen no definida}
	\emptyvarerr{\resumenenv}{#2}{Titulo tesis no definido}
	\emptyvarerr{\resumenenv}{#3}{Titulo seccion no definida}
	\emptyvarerr{\resumenenv}{#4}{Etiqueta marcador del pdf}
	\ifthenelse{\equal{\addabstracttobookmarks}{true}}{
		\phantomsection
		\pdfbookmark{#3}{#4}}{
	}
	\begin{flushright}
		\small
		#1
	\end{flushright}
	\vspace{0.75\baselineskip}
	\begin{center}
		\textcolor{\titlecolor}{\MakeUppercase{\textbf{#2}}}
	\end{center} \newp
	\vspace{-\baselineskip}
	}{%
	\vfill\null
	\clearpage
	\ifthenelse{\equal{\addemptypagespredoc}{true}}{
		\vfill
		\checkoddpage
		\ifoddpage
		\else
			\insertblankpage
		\fi}{
	}
}
\newenvironment{resumen}{
	\begin{resumenenv}{\tablaresumen}{\titulotesis}{\nameabstract}{abstractbookmark}
	}{
	\end{resumenenv}
}
\newenvironment{dedicatoria}{
	\clearpage\null
	\phantomsection
	\vspace{\stretch{1}}
	\begin{flushright}
		\itshape}{
	\end{flushright}
	\vspace{\stretch{2}}
	\null
	\ifthenelse{\equal{\addemptypagespredoc}{true}}{
		\insertblankpage}{
	}
}
\newenvironment{agradecimientos}{
	\clearpage
	\chapter*{\nameagradec}
	\ifthenelse{\equal{\addagradectobookmarks}{true}}{
		\phantomsection
		\pdfbookmark{\nameagradec}{acknowledgments}}{
	}
	~ \\
	\vspace{-2\baselineskip}
	}{
	\clearpage
	\ifthenelse{\equal{\addemptypagespredoc}{true}}{
		\insertblankpage}{
	}
}
\newenvironment{images}[2][]{
	\def\envimageslabelvar {#1}
	\def\envimagescaptionvar {#2}
	\def\GLOBALenvimageinitialized {true}
	\def\GLOBALenvimageadded {false}
	\vspace{\marginimagetop cm}
	\setcaptionmargincm{\captionmarginmultimg}
\begin{figure}[H] \centering
		\thisfloatpagestyle{fancy}%
		\vspace{\marginimagemulttop cm}%
		}{%
		\setcaptionmargincm{\captionlrmargin}%
\ifthenelse{\equal{\envimagescaptionvar}{}}{
			\vspace{\captionlessmarginimage cm}%
		}{%
			\caption{\envimagescaptionvar\envimageslabelvar}%
		}%
	\end{figure}%
	\setcaptionmargincm{\captionlrmargin}
	\vspace{\marginimagebottom cm}
	\def\GLOBALenvimageinitialized {false}
}
\newenvironment{imagesmc}[3][]{
	\def\envimageslabelvar {#1}
	\def\envimagesmcpos {#2}
	\def\envimagescaptionvar {#3}
	\def\GLOBALenvimageinitialized {true}
	\def\GLOBALenvimageadded {false}
	\checkinsidemulticol
\setcaptionmargincm{\captionmarginmultimg}
	\ifthenelse{\equal{#2}{bottom}}{%
\begin{figure*}[hb] \centering
	}{%
	\ifthenelse{\equal{#2}{top}}{%
\begin{figure*}[ht] \centering
	}{%
		\errmessage{LaTeX Warning: Posicion de imagen invalida, valores esperados: bottom,top}
		\stop
	}}%
		\thisfloatpagestyle{fancy}%
		\vspace{\marginimagemulttop cm}%
	}{%
		\setcaptionmargincm{\captionlrmargin}%
\ifthenelse{\equal{\envimagescaptionvar}{}}{
			\vspace{\captionlessmarginimage cm}%
		}{%
			\caption{\envimagescaptionvar\envimageslabelvar}%
		}%
	\end{figure*}%
	\setcaptionmargincm{\captionlrmargin}
	\def\GLOBALenvimageinitialized {false}
}
