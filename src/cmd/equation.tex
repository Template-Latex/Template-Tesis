% Template:     Template Tesis LaTeX
% Documento:    Funciones para insertar ecuaciones
% Versión:      2.3.0 (12/07/2021)
% Codificación: UTF-8
%
% Autor: Pablo Pizarro R.
%        Facultad de Ciencias Físicas y Matemáticas
%        Universidad de Chile
%        pablo@ppizarror.com
%
% Sitio web:    [https://latex.ppizarror.com/tesis]
% Licencia MIT: [https://opensource.org/licenses/MIT]

% Redimensiona una ecuación en linewidth
% 	#1	Tamaño del nuevo objeto (En linewidth)
%	#2	Ecuación a redimensionar
\newcommand{\equationresize}[2]{%
	\emptyvarerr{\equationresize}{#1}{Dimension no definida}%
	\emptyvarerr{\equationresize}{#2}{Ecuacion a redimensionar no definida}%
	\resizebox{#1\linewidth}{!}{$#2$}%
}

% Inserta el caption de un objeto tipo ecuación
%	#1	Texto del caption
\newcommand{\coreinsertequationcaption}[1]{%
	\begin{changemargin}{\captionlrmargin cm}{\captionlrmargin cm}%
		\ifthenelse{\equal{\equationcaptioncenter}{true}}{%
			\centering%
		}{
			\justifying%
		}%
		\textcolor{\captiontextcolor}{%
			\linespread{0.5}\selectfont{%
				\begin{\captionfontsize}#1\end{\captionfontsize}%
			}
		}%
	\end{changemargin}%
}

% Insertar una ecuación
% 	#1	Label (opcional)
%	#2	Ecuación
\newcommand{\insertequation}[2][]{
	\emptyvarerr{\insertequation}{#2}{Ecuacion no definida}
	\ifthenelse{\equal{\numberedequation}{true}}{
		\vspace{\marginequationtop cm}
		\begin{equation}
			\text{#1} #2
		\end{equation}
		\vspace{\marginequationbottom cm}
	}{
		\ifx\hfuzz#1\hfuzz
		\else
			\throwwarning{Label invalido en ecuacion sin numero}
		\fi
		\insertequationanum{#2}
	}
}

% Insertar una ecuación sin número
%	#1	Ecuación
\newcommand{\insertequationanum}[1]{
	\emptyvarerr{\insertequationanum}{#1}{Ecuacion no definida}
	\vspace{\marginequationtop cm}
	\begin{equation*}
		\ensuremath{#1}
	\end{equation*}
	\vspace{\marginequationbottom cm}
}

% Insertar una ecuación en el índice
% 	#1	Label (opcional)
%	#2	Ecuación
%	#3	Título de la ecuación
\newcommand{\insertindexequation}[3][]{
	\emptyvarerr{\insertindexequation}{#2}{Ecuacion no definida}
	\emptyvarerr{\insertindexequation}{#3}{Leyenda no definida}
	\vspace{\margineqnindextop cm}
	\begin{align}
		\text{#1} \ensuremath{#2}
	\end{align}
	\myindexequations{#3}%
	\vspace{\margineqnindexbottom cm}%
	\coreinsertequationcaption{\textit{#3}}%
	\addtocounter{templateIndexEquations}{1}
}

% Insertar una ecuación alineada a la izquierda
% 	#1	Label (opcional)
%	#2	Ecuación
\newcommand{\insertequationleft}[2][]{
	\emptyvarerr{\insertequationleft}{#2}{Ecuacion no definida}
	\ifthenelse{\equal{\numberedequation}{true}}{
		\vspace{\marginequationtop cm}
		\vspace{-\baselineskip}
		\begin{equation}
			\hfilneg \text{#1} #2 \hspace{10000pt minus 1fil}
		\end{equation}
		\vspace{\marginequationbottom cm}
	}{
		\ifx\hfuzz#1\hfuzz
		\else
			\throwwarning{Label invalido en ecuacion sin numero}
		\fi
		\insertequationleftanum{#2}
	}
}

% Insertar una ecuación sin número alineada a la izquierda
%	#1	Ecuación
\newcommand{\insertequationleftanum}[1]{
	\emptyvarerr{\insertequationleftanum}{#1}{Ecuacion no definida}
	\vspace{\marginequationtop cm}
	\vspace{-\baselineskip}
	\begin{equation*}
		\hfilneg \ensuremath{#1} \hspace{10000pt minus 1fil}
	\end{equation*}
	\vspace{\marginequationbottom cm}
}

% Insertar una ecuación alineada a la derecha
% 	#1	Label (opcional)
%	#2	Ecuación
\newcommand{\insertequationright}[2][]{
	\emptyvarerr{\insertequationright}{#2}{Ecuacion no definida}
	\ifthenelse{\equal{\numberedequation}{true}}{
		\vspace{\marginequationtop cm}
		\vspace{-\baselineskip}
		\begin{equation}
			\hspace{10000pt minus 1fil} \text{#1} #2 \hfilneg
		\end{equation}
		\vspace{\marginequationbottom cm}
	}{
		\ifx\hfuzz#1\hfuzz
		\else
			\throwwarning{Label invalido en ecuacion sin numero}
		\fi
		\insertequationrightanum{#2}
	}
}

% Insertar una ecuación sin número alineada a la derecha
%	#1	Ecuación
\newcommand{\insertequationrightanum}[1]{
	\emptyvarerr{\insertequationrightanum}{#1}{Ecuacion no definida}
	\vspace{\marginequationtop cm}
	\vspace{-\baselineskip}
	\begin{equation*}
		\hspace{10000pt minus 1fil} \ensuremath{#1} \hfilneg
	\end{equation*}
	\vspace{\marginequationbottom cm}
}

% Insertar una ecuación con leyenda
% 	#1	Label (opcional)
%	#2	Ecuación
%	#3	Leyenda
\newcommand{\insertequationcaptioned}[3][]{
	\emptyvarerr{\insertequationcaptioned}{#2}{Ecuacion no definida}
	\ifx\hfuzz#3\hfuzz
		\insertequation[#1]{#2}
	\else
		\ifthenelse{\equal{\numberedequation}{true}}{
			\vspace{\marginequationtop cm}
			\begin{equation}
				\text{#1} #2
			\end{equation}
			\vspace{\margineqncaptiontop cm}
			\coreinsertequationcaption{#3}
			\vspace{\margineqncaptionbottom cm}
		}{
			\ifx\hfuzz#1\hfuzz
			\else
				\throwwarning{Label invalido en ecuacion sin numero}
			\fi
			\insertequationcaptionedanum{#2}{#3}
		}
	\fi
}

% Insertar una ecuación con leyenda sin número
%	#1	Ecuación
%	#2	Leyenda
\newcommand{\insertequationcaptionedanum}[2]{
	\emptyvarerr{\insertequationcaptionedanum}{#1}{Ecuacion no definida}
	\ifx\hfuzz#2\hfuzz
		\insertequationanum{#1}
	\else
		\vspace{\marginequationtop cm}
		\begin{equation*}
			\ensuremath{#1}%
		\end{equation*}%
		\vspace{\margineqncaptiontop cm}%
		\coreinsertequationcaption{#2}
		\vspace{\margineqncaptionbottom cm}
	\fi
}

% Insertar una ecuación con el ambiente gather
%	#1	Ecuación
\newcommand{\insertgather}[1]{
	\emptyvarerr{\insertgather}{#1}{Ecuacion no definida}
	\ifthenelse{\equal{\numberedequation}{true}}{
		\vspace{\margingathertop cm}
		\begin{gather}
			\ensuremath{#1}
		\end{gather}
		\vspace{\margingatherbottom cm}
	}{
		\insertgatheranum{#1}
	}
}

% Insertar una ecuación con el ambiente gather sin número
%	#1	Ecuación
\newcommand{\insertgatheranum}[1]{
	\emptyvarerr{\insertgatheranum}{#1}{Ecuacion no definida}
	\vspace{\margingathertop cm}
	\begin{gather*}
		\ensuremath{#1}
	\end{gather*}
	\vspace{\margingatherbottom cm}
}

% Insertar una ecuación (gather) con leyenda
%	#1	Ecuación
%	#2	Leyenda
\newcommand{\insertgathercaptioned}[2]{
	\emptyvarerr{\insertgathercaptioned}{#1}{Ecuacion no definida}
	\ifx\hfuzz#2\hfuzz
		\insertgather{#1}
	\else
		\ifthenelse{\equal{\numberedequation}{true}}{
			\vspace{\margingathertop cm}
			\begin{gather}
				\ensuremath{#1}
			\end{gather}
			\vspace{\margingathercapttop cm}
			\coreinsertequationcaption{#2}
			\vspace{\margingathercaptbottom cm}
		}{
			\insertgathercaptionedanum{#1}{#2}
		}
	\fi
}

% Insertar una ecuación (gather) con leyenda sin número
%	#1	Ecuación
%	#2	Leyenda
\newcommand{\insertgathercaptionedanum}[2]{
	\emptyvarerr{\insertgathercaptionedanum}{#1}{Ecuacion no definida}
	\ifx\hfuzz#2\hfuzz
		\insertgatheranum{#1}
	\else
		\vspace{\margingathertop cm}
		\begin{gather*}
			\ensuremath{#1}
		\end{gather*}
		\vspace{\margingathercapttop cm}
		\coreinsertequationcaption{#2}
		\vspace{\margingathercaptbottom cm}
	\fi
}

% Insertar una ecuación con el ambiente gathered
% 	#1	Label (opcional)
%	#2	Ecuación
\newcommand{\insertgathered}[2][]{
	\emptyvarerr{\insertgathered}{#2}{Ecuacion no definida}
	\ifthenelse{\equal{\numberedequation}{true}}{
		\vspace{\marginequationtop cm}
		\begin{equation}
			\begin{gathered}
				\text{#1} \ensuremath{#2}
			\end{gathered}
		\end{equation}
		\vspace{\margingatheredbottom cm}
	}{
		\ifx\hfuzz#1\hfuzz
		\else
			\throwwarning{Label invalido en ecuacion (gathered) sin numero}
		\fi
		\vspace{\margingatheredtop cm}
		\begin{gather*}
			\ensuremath{#2}
		\end{gather*}
		\vspace{\margingatheredbottom cm}
	}
}

% Insertar una ecuación con el ambiente gathered sin número
%	#1	Ecuación
\newcommand{\insertgatheredanum}[1]{
	\emptyvarerr{\insertgatheredanum}{#1}{Ecuacion no definida}
	\vspace{\margingatheredtop cm}
	\begin{gather*}
		\ensuremath{#1}
	\end{gather*}
	\vspace{\margingatheredbottom cm}
}

% Insertar una ecuación (gathered) con leyenda
% 	#1	Label (opcional)
%	#2	Ecuación
%	#3	Leyenda
\newcommand{\insertgatheredcaptioned}[3][]{
	\emptyvarerr{\insertgatheredcaptioned}{#2}{Ecuacion no definida}
	\ifx\hfuzz#3\hfuzz
		\insertgathered[#1]{#2}
	\else
		\ifthenelse{\equal{\numberedequation}{true}}{
			\vspace{\marginequationtop cm}
			\begin{equation}
				\begin{gathered}
					\text{#1} \ensuremath{#2}
				\end{gathered}
			\end{equation}
			\vspace{\margingatheredcapttop cm}
			\coreinsertequationcaption{#3}
			\vspace{\margingatheredcaptbottom cm}
		}{
			\ifx\hfuzz#1\hfuzz
			\else
				\throwwarning{Label invalido en ecuacion (gathered) sin numero}
			\fi
			\insertgatheredcaptionedanum{#2}{#3}
		}
		\fi
}

% Insertar una ecuación (gathered) con leyenda sin número
%	#1	Ecuación
%	#2	Leyenda
\newcommand{\insertgatheredcaptionedanum}[2]{
	\emptyvarerr{\insertgatheredcaptionedanum}{#1}{Ecuacion no definida}
	\ifx\hfuzz#2\hfuzz
		\insertgatheredanum{#1}
	\else
		\vspace{\margingatheredtop cm}
		\begin{gather*}
			\ensuremath{#1}
		\end{gather*}
		\vspace{\margingatheredcapttop cm}
		\coreinsertequationcaption{#2}
		\vspace{\margingathercaptbottom cm}
	\fi
}

% Insertar una ecuación con el ambiente align
%	#1	Ecuación
\newcommand{\insertalign}[1]{
	\emptyvarerr{\insertalign}{#1}{Ecuacion no definida}
	\ifthenelse{\equal{\numberedequation}{true}}{
		\vspace{\marginaligntop cm}
		\begin{align}
			\ensuremath{#1}
		\end{align}
		\vspace{\marginalignbottom cm}
	}{
		\insertalignanum{#1}
	}
}

% Insertar una ecuación con el ambiente align sin número
%	#1	Ecuación
\newcommand{\insertalignanum}[1]{
	\emptyvarerr{\insertalignanum}{#1}{Ecuacion no definida}
	\vspace{\marginaligntop cm}
	\begin{align*}
		\ensuremath{#1}
	\end{align*}
	\vspace{\marginalignbottom cm}
}

% Insertar una ecuación (align) con leyenda
%	#1	Ecuación
%	#2	Leyenda
\newcommand{\insertaligncaptioned}[2]{
	\emptyvarerr{\insertaligncaptioned}{#1}{Ecuacion no definida}
	\ifx\hfuzz#2\hfuzz
		\insertalign{#1}
	\else
		\ifthenelse{\equal{\numberedequation}{true}}{
			\vspace{\marginaligntop cm}
			\begin{align}
				\ensuremath{#1}
			\end{align}
			\vspace{\marginaligncapttop cm}
			\coreinsertequationcaption{#2}
			\vspace{\marginaligncaptbottom cm}
		}{
			\insertaligncaptionedanum{#1}{#2}
		}
	\fi
}

% Insertar una ecuación (align) con leyenda sin número
%	#1	Ecuación
%	#2	Leyenda
\newcommand{\insertaligncaptionedanum}[2]{
	\emptyvarerr{\insertaligncaptionedanum}{#1}{Ecuacion no definida}
	\ifx\hfuzz#2\hfuzz
		\insertalignanum{#1}
	\else
		\vspace{\marginaligntop cm}
		\begin{align*}
			\ensuremath{#1}
		\end{align*}
		\vspace{\marginaligncapttop cm}
		\coreinsertequationcaption{#2}
		\vspace{\marginaligncaptbottom cm}
	\fi
}

% Insertar una ecuación con el ambiente aligned
% 	#1	Label (opcional)
%	#2	Ecuación
\newcommand{\insertaligned}[2][]{
	\emptyvarerr{\insertaligned}{#2}{Ecuacion no definida}
	\ifthenelse{\equal{\numberedequation}{true}}{
		\vspace{\marginequationtop cm}
		\begin{equation}
			\begin{aligned}
				\text{#1} \ensuremath{#2}
			\end{aligned}
		\end{equation}
		\vspace{\marginalignedbottom cm}
	}{
		\ifx\hfuzz#1\hfuzz
		\else
			\throwwarning{Label invalido en ecuacion (aligned) sin numero}
		\fi
		\insertalignedanum{#2}
	}
}

% Insertar una ecuación con el ambiente aligned sin número
%	#1	Ecuación
\newcommand{\insertalignedanum}[1]{
	\emptyvarerr{\insertalignedanum}{#1}{Ecuacion no definida}
	\vspace{\marginalignedtop cm}
	\begin{align*}
		\ensuremath{#1}
	\end{align*}
	\vspace{\marginalignedbottom cm}
}

% Insertar una ecuación (aligned) con leyenda
% 	#1	Label (opcional)
%	#2	Ecuación
%	#3	Leyenda
\newcommand{\insertalignedcaptioned}[3][]{
	\emptyvarerr{\insertalignedcaptioned}{#2}{Ecuacion no definida}
	\ifx\hfuzz#3\hfuzz
		\insertaligned[#1]{#2}
	\else
		\ifthenelse{\equal{\numberedequation}{true}}{
			\vspace{\marginequationtop cm}
			\begin{equation}
				\begin{aligned}
					\text{#1} \ensuremath{#2}
				\end{aligned}
			\end{equation}
			\vspace{\marginalignedcapttop cm}
			\coreinsertequationcaption{#3}
			\vspace{\marginalignedcaptbottom cm}
		}{
			\ifx\hfuzz#1\hfuzz
			\else
				\throwwarning{Label invalido en ecuacion (aligned) sin numero}
			\fi
			\insertalignedcaptionedanum{#2}{#3}
		}
	\fi
}

% Insertar una ecuación (aligned) con leyenda sin número
%	#1	Ecuación
%	#2	Leyenda
\newcommand{\insertalignedcaptionedanum}[2]{
	\emptyvarerr{\insertalignedcaptionedanum}{#1}{Ecuacion no definida}
	\ifx\hfuzz#2\hfuzz
		\insertalignedanum{#1}
	\else
		\vspace{\marginequationtop cm}
		\begin{equation}
			\begin{aligned}
				\ensuremath{#1}
			\end{aligned}
		\end{equation}
		\vspace{\marginalignedcapttop cm}
		\coreinsertequationcaption{#2}
		\vspace{\marginalignedcaptbottom cm}
	\fi
}
