% Template:     Template Tesis LaTeX
% Documento:    Funciones del núcleo del template
% Versión:      2.3.1 (17/07/2021)
% Codificación: UTF-8
%
% Autor: Pablo Pizarro R.
%        Facultad de Ciencias Físicas y Matemáticas
%        Universidad de Chile
%        pablo@ppizarror.com
%
% Sitio web:    [https://latex.ppizarror.com/tesis]
% Licencia MIT: [https://opensource.org/licenses/MIT]

% Definición de variables globales
\def\GLOBALcaptiondefn {EMPTY-VAR}       % Definición del caption
%\def\GLOBALcaptiondefnimages{EMPTY-VAR} % Definición del caption en images
%\def\GLOBALcaptiondefnsrc{EMPTY-VAR}    % Definición del caption en sourcecode
\def\GLOBALchapternumenabled {false}     % Numeración de capítulos empezó
\def\GLOBALenvimageadded {false}         % Indica que una imagen ha sido añadida
\def\GLOBALenvimageinitialized {false}   % Entorno images activo
\def\GLOBALenvmulticol {false}           % Indica que el entorno multicol está activo
\def\GLOBALsectionalph {false}           % Sección con numeración de letras
\def\GLOBALsectionanumenabled {false}    % Sección sin numeración
\def\GLOBALsubsectionanumenabled {false} % Subsección sin numeración
\def\GLOBALtablerowcolorindex {2}        % Índice tabla colores
\def\GLOBALtablerowcolorswitch {false}   % Tabla con colores cambiados

% Contador global de objetos
\newcounter{templateEquations}      % Ecuaciones
\newcounter{templateFigures}        % Figuras
\newcounter{templateIndexEquations} % Ecuaciones en el índice
\newcounter{templateListings}       % Códigos fuente
\newcounter{templatePageCounter}    % Administra números de páginas
\newcounter{templateTables}         % Tablas

% Contador nivel de bookmarks marcadores
\newcounter{templateBookmarksLevelPrev}
\setcounter{templateBookmarksLevelPrev}{\cfgbookmarksopenlevel}
\addtocounter{templateBookmarksLevelPrev}{-1}

% Otros
\let\latex\LaTeX

% Lanza un mensaje de error
% 	#1	Función del error
%	#2	Mensaje
\newcommand{\throwerror}[2]{%
	\errmessage{LaTeX Error: \noexpand#1 #2 (linea \the\inputlineno)}%
	\stop
}

% Lanza un mensaje de advertencia
%	#1	Mensaje
\newcommand{\throwwarning}[1]{%
	\errmessage{LaTeX Warning: #1 (linea \the\inputlineno)}%
}

% Lanza un mensaje de error indicando mala configuración dentro de begin{document}
%	#1	Mensaje de error
% 	#2	Configuración usada
%	#3	Valores esperados
\newcommand{\throwbadconfigondoc}[3]{%
	\errmessage{#1 \noexpand #2=#2. Valores esperados: #3}%
	\stop
}

% Comprueba si una variable está definida
%	#1	Variable
\newcommand{\checkvardefined}[1]{%
	\ifthenelse{\isundefined{#1}}{%
		\errmessage{LaTeX Warning: Variable \noexpand#1 no definida}%
		\stop}{%
	}%
}

% Escribe un mensaje en la consola
%	#1	Mensaje
\newcommand{\coretemplatemessage}[1]{%
	\message{Template: #1}%
}

% Comprueba si una variable está definida
%	#1	Variable
%	#2	Mensaje
\newcommand{\checkextravarexist}[2]{%
	\ifthenelse{\isundefined{#1}}{%
		\errmessage{LaTeX Warning: Variable \noexpand#1 no definida}%
		\ifx\hfuzz#2\hfuzz%
			\errmessage{LaTeX Warning: Defina la variable en el bloque de INFORMACION DEL DOCUMENTO al comienzo del archivo principal del template}%
		\else%
			\errmessage{LaTeX Warning: #2}%
		\fi}{%
	}
}

% Lanza un mensaje de error si una variable no ha sido definida
% 	#1	Función del error
%	#2	Variable
%	#3	Mensaje
\newcommand{\emptyvarerr}[3]{%
	\ifx\hfuzz#2\hfuzz%
		\errmessage{LaTeX Warning: \noexpand#1 #3 (linea \the\inputlineno)}%
	\fi
}

% Cambiar el margen de los caption
% 	#1	Margen en centímetros
\newcommand{\setcaptionmargincm}[1]{
	\captionsetup{margin=#1cm}
}

% Cambia márgenes de las páginas [cm]
% 	#1	Margen izquierdo
%	#2	Margen superior
%	#3	Margen derecho
%	#4	Margen inferior
\newcommand{\setpagemargincm}[4]{
	\ifthenelse{\equal{\compilertype}{lualatex}}{
		% Geometry no válido en lualatex
	}{
		\newgeometry{left=#1cm, top=#2cm, right=#3cm, bottom=#4cm}
	}
}

% Define el caption del índice
% 	#1	Título del caption
\newcommand{\setindexcaption}[1]{\def\GLOBALcaptiondefn {#1}}
%\newcommand{\setindexcaptionimages}[1]{\def\GLOBALcaptiondefnimages {#1}}
%\newcommand{\setindexcaptionsourcecode}[1]{\def\GLOBALcaptiondefnsrc {#1}}

% Resetea los caption
\newcommand{\resetindexcaption}{\def\GLOBALcaptiondefn {EMPTY-VAR}}
%\newcommand{\resetindexcaptionimages}{\def\GLOBALcaptiondefnimages {EMPTY-VAR}}
%\newcommand{\resetindexcaptionsourcecode}{\def\GLOBALcaptiondefnsrc {EMPTY-VAR}}

% Cambia los márgenes del documento
%	#1	Margen izquierdo
%	#2	Margen derecho
\newcommand{\changemargin}[2]{%
	\emptyvarerr{\changemargin}{#1}{Margen izquierdo no definido}%
	\emptyvarerr{\changemargin}{#2}{Margen derecho no definido}%
	\list{}{\rightmargin#2\leftmargin#1}\item[]%
}
\let\endchangemargin=\endlist

% Chequea que las funciones sólo puedan usarse en el entorno images
\newcommand{\checkonlyonenvimage}{%
	\ifthenelse{\equal{\GLOBALenvimageinitialized}{true}}{}{%
		\throwwarning{Funciones \noexpand\addimage o \noexpand\addimageboxed no pueden usarse fuera del entorno \noexpand\images}\stop%
	}%
}

% Chequea que las funciones sólo puedan usarse fuera del entorno images
\newcommand{\checkoutsideenvimage}{%
	\ifthenelse{\equal{\GLOBALenvimageinitialized}{true}}{%
		\throwwarning{Esta funcion solo puede usarse fuera del entorno \noexpand\images}%
		\stop}{%
	}%
}

% Chequea que las funciones puedan usarse solo en el entorno multicol
\newcommand{\checkinsidemulticol}{%
	\ifthenelse{\equal{\GLOBALenvmulticol}{false}}{%
		\throwwarning{Esta funcion solo puede usarse dentro de multicols}%
		\stop}{%
	}
}

% Verifica que una variable sea del estilo "true" o "false"
\newcommand{\corecheckbooleanvar}[1]{%
	\emptyvarerr{\corecheckbooleanvar}{#1}{Variable no definida}%
	\ifthenelse{\equal{#1}{true}}{}{%
		\ifthenelse{\equal{#1}{false}}{}{%
			\throwwarning{Variable debe ser true o false}\stop}%
	}%
}

% Centra verticalmente un texto
%	#1	Texto a centrar
\newcommand{\verticallycentertext}[1]{%
	\emptyvarerr{\verticallycentertext}{#1}{Centra un texto verticalmente}%
	\topskip0pt%
	\vspace*{\fill}%
	#1%
	\vspace*{\fill}%
}

% Agrega una carpeta al path de imágenes
%	#1	Carpeta
\makeatletter
\newcommand\addpathimage[1]{\gappto\Ginput@path{{#1}}}
\makeatother

% Verifica que un tamaño de fuente sea correcto
%	#1	Tamaño de fuente
\newcommand{\corecheckfontsize}[1]{%
	\ifthenelse{\equal{#1}{normalsize}}{}{%
	\ifthenelse{\equal{#1}{small}}{}{%
	\ifthenelse{\equal{#1}{large}}{}{%
	\ifthenelse{\equal{#1}{Large}}{}{%
	\ifthenelse{\equal{#1}{LARGE}}{}{%
	\ifthenelse{\equal{#1}{huge}}{}{%
	\ifthenelse{\equal{#1}{Huge}}{}{%
	\ifthenelse{\equal{#1}{HUGE}}{}{%
	\ifthenelse{\equal{#1}{footnotesize}}{}{%
	\ifthenelse{\equal{#1}{scriptsize}}{}{%
	\ifthenelse{\equal{#1}{tiny}}{}{%
		\errmessage{LaTeX Warning: Tamano de fuente incorrecto (\noexpand #1= #1). Valores esperados: tiny,scriptsize,footnotesize,small,normalisize,large,Large,LARGE,huge,Huge,HUGE}%
		\stop%
		}}}}}}}}}}%
	}%
}
